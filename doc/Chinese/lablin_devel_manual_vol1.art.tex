\ifx \inmaxwitbook \undefined
\documentclass[a4paper,11pt]{article}

\include{style}


\setlength{\parindent}{2em}

\begin{CJK*}{UTF8}{gbsn}
\title{g-bios开发者手册(第1卷:使用入门)}
\author{MaxWit魔鬼训练营}
\end{CJK*}

%\renewcommand{\baselinestretch}{1.3}
\linespread{1.3}

\begin{document}
\AtBeginDvi{\special{pdf:tounicode UTF8-UCS2}}
\begin{CJK*}{UTF8}{gbsn}
\CJKtilde
\maketitle
\tableofcontents

\fi

\clearpage
%\chapter{Embedded System and Embedded OS}
\section{Labin~体系结构}
\subsection{Lablin~介绍}
\subsection{Lablin~体系结构}

WinCE, WebOS, BlackBerry, iOS, Linux, Android, Bada, meego
\section{~X86~体系结构}
\subsection{~Host and target~}
\begin{enumerate}
\item 开发板~(target)~与~PC~机~(host)~之间的连接线/通信
	\begin{enumerate}
	\item 网线~(RJ45)
	\item USB~(可充当网线)
		\begin{enumerate}\setlength{\itemsep}{-\itemsep}
		\item Usb device(power, Mass starge, Usb modem, NIC)
		\item Usb host
		\end{enumerate}
	\item JTAG
	\item 串口~(RS232)URAT
	\item WiFi
	\item Bluetooth
	\end{enumerate}
\end{enumerate}
\subsection{~Host~开发环境}
\begin{enumerate}
\item Minicom/kermit 
\item Nfs/tftp 
\item ADS/XDB(pxa168)/OpenOCD$\rightarrow$GDB
\item Mount, ppp(modem)
\item GNU Toolchain(cross)
\item qemu
\end{enumerate}

% \section{~Host~端发行版的选择及软件安装}
% \subsection{关于~Linux~发行版}
% 目前已测试通过的发行版有(包括64位版):~Debian 5.0~、~Ubuntu 9.04~、~Ubuntu 8.10~、~Fedora Core 10~,推荐使用~Debian 5.0~。若有人有兴趣测试并支持其他~Linux~发行版,欢迎把~patch~发给~Maxwit~项目维护者:
% 
% \begin{tabular}[t]{|l|c|}
% \hline
% 维护者 & 邮箱 \\
% \hline
% Conke Hu & conke@maxwit.com \\
% Tiger Yu & tiger@maxwit.com \\
% Fleya Hou & fleya@maxwit.com \\
% \hline
% \end{tabular}
% 
% \subsection{安装软件包}
% 必须安装的软件包:
% 
% ~gcc~,g++,make,subversion,git-core,tftpd-hpa,tftp-hpa,nfs-kernel-server,libsdl及其dev包,autoconf,automake,m4,libtool,
% 
% 64位系统上需要额外安装的软件包:
% 
% ~libc6-dev-i386~
% 
% ~Debian~或~Ubuntu~系统上可通过如下命令安装有软件包:
% \begin{lstlisting}[language=bash,numbers=none]
% $apt-get install gcc g++ make subversion git-core libncures5-dev zliblg-dev libsdl1.2-dev tftpd-hpa nfs-kernel-server autoconf automake m4 libtool
% \end{lstlisting}
% 注:Ubuntu用户还需执行以下操作
% \begin{lstlisting}[language=bash,numbers=none]
% $dpkg-reconfigure dash
% \end{lstlisting}
% 然后选择``no'',执行ls -l ~/bin/sh~确认已指向~/bin/bash~。\\
% 
% \section{~Host~ 端设置}
% 
% \subsection{安装~NFS~ ~Server~}
% 
% \begin{lstlisting}[language=bash,numbers=none]
% $apt-get install nfs-kerne-server 
% \end{lstlisting}
% 
% 第二步,编辑/etc/exports文件,添加下面两行:
% 
% /root/maxwit/rootfs  *(rw, sync, no\_root\_squash,no\_subtree\_check)
% 
% 第三步,重启~NFS~ ~Server~:
% 
% \# /etc/init.d/nfs-kernel-server restart
% 
% 第四步,测试~NFS~ ~Server~:
% 
% \# ~mount~ -t ~nfs~ 192.168.0.111:/root/maxwit/rootfs /mnt/
% 
% (假定本机IP为192.168.0.111)
% 
% \subsection{安装Kermit}
% 
% 第一步,从源码安装~Kermit~(若前面已安装~kermit~,则略过这一步)
% 
% ~make~ ~linux~ \&\& ~make~ ~install~
% 
% ~wget~ http://maxwit.googlecode.com/files/kermrc
% 
% cp -v kermrc  $\sim$/.kermrc
% 
% 第二步,打开$\sim$/.kermrc,修改``set line''一行,确认你所用的串口设备,若用的是USB-to-Serial转接器,可以改成:``set line /dev/ttyUSB0''
% 
% \subsection{安装TFTP Server}
% \
% 
% 第一步,编译tftp软件(如果前面已经通过apt方式安装了tftp,则跳过这一步)
% 
% 
% \# tar jxvf tftp-hpa-0.40.tar.bz2 
% 
% \#./configure --prefix=/usr
% 
% \# ~make~ \&\& ~make~ ~install~
% 
% 第二步,更改~tftpd~下载目录
% ~tftp~服务器的默认下载目录是/var/lib/tftpboot,我们要改为\$\{HOME\}/maxwit/images.
% 
% 打开/etc/inetd.conf,找到以``tftpd''开头的一行,将其中的/var/lib/tftpboot改为
% t 方式
% 
% \$\{HOME\}/maxwit/images:
% 
% ~tftp~   ~dgram~    ~udp~  ~wait~   ~root~    /usr/sbin/in.tftpd   /usr/sbin/in.tftpd --S
% 
% \$\{HOME\}/maxwit/images
% 
% 第三步,/etc/init.d/tftpd-hpa ~restart~
% 
% 第四步,测试~tftp~ ~server~
% 
% \# ~cd~ /~tmp~
% 
% \# ~echo~ ~hello~ $>$ ~/images/test
% 
% \# ~chmod~ 666 ~/images/test
% 
% \# ~tftp~ 192.168.0.111(假定本机IP为192.168.0.111)
% 
% $>$~get~ ~test~
% 
% $>$~quit~
% 
% \# ~cat~ ~test~
% 
% \# rm test ~/images/test
% 

\section{使用~Powertool~源码}
\section{使用~Lablin~源码}
\subsection{获取~Lablin~最新源码}
在MaxWit~开放实验室的开源项目主页(http://maxwit.googlecode.com)~的``Source''~页面上可以下载到全部源码。google~提供的默认下载方式是: \\
\begin{lstlisting}[language=bash,numbers=none]
$ cd
$ git clone git://github.com/maxwit/lablin.git
\end{lstlisting}

\subsection{~Lablin~源码目录介绍}

\begin{equation*}
	\text{Lablin}
    \left\{
        \begin{aligned}
            \text{build$\rightarrow$}&
			\left\{
				\begin{aligned}
				\text{Lablin building Menu}\\ 
				\end{aligned}
			\right.\\
            \text{core} &
            \left\{
                \begin{aligned}
                    \text{bmw\_base}\\
                    \text{bmw\_menu}\\
					\text{bmw\_pkgs}
                \end{aligned}
            \right\}
			\text{公共环境变量和函数host~端设置}\\
            \text{target} &
            \left\{
                \begin{aligned}
                    \text{build}\\
                    \text{DirectFB-1.4.3} &
					\left\{
						\begin{aligned}
							\text{build.sh}\\
							\text{dep}\\
							\text{fix\_for\_libpng-1.4.2.patch}
						\end{aligned}
					\right.\\
					\text{...}\\
                \end{aligned}
            \right\}
			\text{目标板环境的搭建}\\
            \text{doc} &
            \left\{
                \begin{aligned}
                    \text{Chinese}\\
                    \text{labin\_arch.jpg}
                \end{aligned}
            \right\}
			\text{Lablin~使用文档}
        \end{aligned}
    \right.
\end{equation*}

% \begin{numcases}{Lablin} 
% build & Lablin building Menu\\
% core & 公共环境变量和函数host~端设置\\
% target &目标板环境的搭建\\
% doc & Lablin~使用文档
% \end{numcases} 

执行build~文件,即进入~Lalin Building Menu~。
\noindent~Lablin Building Menu~(如下):\\
\begin{verbatim}
[MaxWit Lablin Building Menu] (for AT91SAM9263)

     1). Build Basic System (linux-2.6.29.4 & busybox)
     2). Build Applications (Lib/App/Games)
     3). Build Qtopia
     4). Testing on QEMU
     5). Create File System Images (UBI/YAFFS2/JFFS2,etc.)
     x). Exit

Your choice[1-5]? 
\end{verbatim}
\noindent以下是各选项的详解:\\
\begin{table}[htbp]
\begin{tabular}{|p{5cm}|p{6cm}|}
\hline
\textbf{Options} & \textbf{Function Description} \\ \hline
Build Basic System(linux-2.6.29.4 \& busybox) & 默认编译~realview~的~kernel~和一个基本的系统。\\ \hline
Build Applications(Lib/App/Games) & 编译相关依赖库,应用程序。\\ \hline
Build Qtopia & 编译桌面系统~Qtopia~。\\ \hline
Testing on QEMU & 用~qemu~测试编译好的系统(注:编译生成的文件系统在\$HOME/maxwit/rootfs~目录下)。\\ \hline
Create File System Images(UBI/YAFFS2/JFFS2,etc.) & 创建~rootfsimage~,即将\$HOME/maxwit/rootfs~目录下做成各种文件类型的~image~,并存放到\$HOME/maxwit/images~目录中。 \\ \hline
\end{tabular}
\end{table}
\pagebreak[4]


\subsection{~Lablin~生成目录介绍}
~Lablin~编译后完成后,会在~\$HOME~目录下生成~maxwit~文件夹,~maxwit~文件中目录结构如下所示:
\begin{numcases}{maxwit}
build & 源码包编译的地方\\
images & 存放各种image(kernel image、bootloader image、rootfs image)\\
rootfs & 根文件系统\\
source & 存放源码包的地方。如果存在~/work/source~,则该文件夹不存在 \\
toolchain & 交叉编译工具\\
utils & 编译过程中所需工具
\end{numcases}
\subsection{安装~Toolchain~}
见PowerTool~相关文档。

\subsection{编译~Lablin~基本系统}
\begin{lstlisting}[language=sh,numbers=none]
$ cd ~/lablin
$ ./build
\end{lstlisting}
\noindent选择
\begin{verbatim}
1). Build Basic System (linux-2.6.29.4 & busybox)
\end{verbatim}

至少要编译``Basic System''~,这样一个基本的嵌入式~Linux~系统就能跑起来,但想要加入丰富的软件,还要继续编译``Libraries''、``Applications''和``Game''等其他模块(第一次执行时,脚本会自动下载所需的第三方源码包。)

再依次选择2、4,看到小企鹅和~console~了吗?尽情玩吧,目前~Lablin~里已经有不少有趣的小东东了。当然,最有趣也更重要的是,我们一起参与开发,一起来完善她!

\section{模拟器中运行Lablin}
\subsection{QEMU安装}

\subsubsection{安装}
\noindent
~qemu~的安装及参数详解
\begin{enumerate}
\item 获取~qemu~源码(撰写本文时~qemu~最新为0.10.2).
\begin{lstlisting}[language=sh, numbers=none]
$ wget http://???
$ tar jxvf qemu-0.10.2.tar.bz2
\end{lstlisting}
\item Configuration
\begin{lstlisting}[language=sh, numbers=none]
$ cd qemu-0.10.2
$ ./configure --prefix=/usr --target-list=arm-softmmu,mips-softmmu,mipsel-soft-mmu
\end{lstlisting}
\item 编译并安装
\begin{lstlisting}[language=sh, numbers=none]
$ make
$ make install 
\end{lstlisting}
将在~/usr/bin/~目录下生成如下程序:
\end{enumerate}

\subsubsection{~QEMU~ ~for~ ~MIPS~}
\begin{enumerate}
\item 项目描述

不需要硬件平台

\item 系统要求

SDL~开发库

losetup~工具

\item ~qemu~命令行参数

~qemu-system-mipsel~:

-M malta...

\item 编译~Toolchain~

参照脚本

使用最新版~gcc!~

\item 编译~Mips Linux~内核
\begin{lstlisting}[language=sh, numbers=none]
$ make ARCH=mips_makta_defconfig
$ make ARCh=mips_CROSS_COMPOILE=mipsel-linux-
$ make
\end{lstlisting}
使用qemu测试你的内核

\begin{lstlisting}[language=sh, numbers=none]
$ qemu-systerm-mipsel -M malta -kernek vmlinux -serial stdio -append "root=/dev/hda1 console=ttyS0"
\end{lstlisting}

\item 制作~Hard Disk image~

\begin{lstlisting}[language=sh, numbers=none]
$ dd if=/dev/zero of=rfs,img bs=1024 count=16k
$ fdisk -C 128 rfs,img
\end{lstlisting}


N $\rightarrow$ p $\rightarrow$ q
\begin{lstlisting}[language=sh, numbers=none]
$ losetup -O 32256 /dev/loop1 rfs,img
$ mkfs.ext4 /dev/loop1
$ mount /dev/loop1 /mnt/1
\end{lstlisting}
测试~HD image~
\begin{lstlisting}[language=sh, numbers=none]
$ qemu-system-mipsel -M malta -Kernel vmlinux -had rfs.img
\end{lstlisting}
\item 创建~rootfs~

Busybox

Copy libc

\item 运行整个~MIPS Linux~

Console

tty

\item 完善
rcS
mount /proc
mount /sys
mdev/udev
\end{enumerate}

\subsubsection{QEMU for ARM}

\section{运行~Lablin~(基于实际硬件平台)}
\subsection{编译~Bootloader~}

这里,我们使用~g-bios~作为~Lablin~的~bootloader~,当然,你也可以使用其他的~bootloader~,但我们认为~g-bios~跟强,更方便,可以提高整个工作效率。

BTW~,我们以~ATMEL AT91SAM9263~为示例硬件平台,你可以使用~S3C24XX~或其他的平台。另外,为了简化,目前的这个文档只介绍~NFS root~方式启动,当然,~Native~方式(直接从~Flash~启动)也支持得很好,你可以自己试试。

第一步,配置
\begin{lstlisting}[language=c,numbers=none]
$ cd ~/g-bios
$ make **_defconfig
\end{lstlisting}
Platform~选择~AT91SAM9263~

第二步,编译
\begin{lstlisting}[language=c,numbers=none]
$ make
\end{lstlisting}
会生成~g-bios-th.bin~和~g-bios-bh.bin~并已自动~copy~到$\sim$/maxwit/images~下,以备后继下载和烧录。(g-bios-th.bin~和~g-bios-bh.bin~分别对应~g-bios~上半部和下半部。)

\subsection{编译~Linux Kernel}
第一步,解压~Linux Kernel
\begin{lstlisting}[language=c,numbers=none]
$ tar jxf linux-2.6.28.tar.bz2
$ cd linux-2.6.28
\end{lstlisting}

第二步,配置~Kernel

\begin{lstlisting}[language=c,numbers=none]
$ make ARCH=arm at91sam9263ek_defconfig
$ make ARCH=arm menuconfig
\end{lstlisting}

然后编译以下几个选项:\\
(1)``Kernel Features''$\rightarrow$选上~EABI~选项,并去掉~OABI!~\\
(2)``Network file system''$\rightarrow$选上``NFS client''~和~``Root file system on NFS''~保存,退出

第三步,编译
\begin{lstlisting}[language=c,numbers=none]
$ make ARCH=arm CROSS_COMPILE=arm-maxwit-linux-gnueabi- --j4
$ cp -v arch/arm/boot/zImage ~/maxwit/images
\end{lstlisting}

\subsection{烧录~images~}
第一步,接上~USB~线和网线.

第二步,先将跳线拔到~off~位置,然后上电!
\begin{lstlisting}[language=c,numbers=none]
$ lsusb
\end{lstlisting}
你将会看到``Atmel....''~的字样,否则,先断电,然后重做第一步和第二步。

第三步,安装开发板驱动。

首先删除所有~usbserial~模块:

\begin{lstlisting}[language=c,numbers=none]
$ lsmod | grep \^usbserial
$ rmmod xxx
$ rmmod usbserial
\end{lstlisting}

然后安装开发板驱动:
\begin{lstlisting}[language=c,numbers=none]
$ modprobe usbserial vendor=0x03eb product=0x6124
\end{lstlisting}
查看~USB~设备是否创建:

\begin{lstlisting}[language=c,numbers=none]
$ ls/dev/ttyUSB*
\end{lstlisting}

第四步,安装并运行~SAM-BA~

\begin{lstlisting}[language=c,numbers=none]
$ unzip sam-ba_cdc_2.7.linux_01.zip
$ cd sam-ba_cde_2.7.linux_01
$ ./sam-ba_cdc_2.7.linux_01
\end{lstlisting}

选择~AT91SAM9263-EK~,然后点击~``Connect''~

第五步,烧写~g-bios~的上半部。

将跳线拨到~on~位置,然后执行下列操作
``Enable Dataflash''$\rightarrow$Execute

``Erase All''$\rightarrow$Execute

``Send Boot File''$\rightarrow$Execute$\rightarrow$Select g-bios-th.bin.OK

第六步,烧写~g-bios~下半部。

可参考~g-bios~开发者手册烧录~g-bios-bh.bin~
再按下开发板~Reset~键,然后选择1(从~Flash~中启动),回车,进入~g-bios~命令行。

\subsection{启动~Lablin~}
如前所述,但为了方便起见,我们先用~TFTP + NFS~方式启动~Linux~。
在~g-bios~命令行下,输入:

\begin{lstlisting}[language=c,numbers=none]
$ boot -t zImage -n
\end{lstlisting}

[说明]

$>$ -t [filename]:用~tftp~方式下载指定的~kernel image~。

$>$ -n [nfs\_server:/nfs/path/]:用~NFS~方式~mount rootfs~。也可以加上参数,如:
``-n 192.168.0.111:/path/to/nfs''。

$>$ ~boot~程序具有记忆功能,即能记住用户输入的参数,换句话说,再次输入~boot~时不再,需要输入参数了,除非你想重设参数。

现在看到~``maxwit login:''~的登陆提示符了吗?

用~root~用户登陆,输入~root~回车,进入系统~Lablin~里已经有不少有趣的小东东了。~Enjoy yourself now!~

这个文档可能有疏漏的地方,但实际的操作步骤我们已经测过多遍,所以一定能跑起来,若有任何疑问,可以在~ChinaUnix~的~MaxWit~版块上发贴讨论,或直接给我们发~mail~。不怕路远的朋友还可直接来上海的~MaxWit~开放实验室~face-to-face~地交流!
\section{体验~Lablin~}
已支持:

不支持:蓝牙

\subsection{图片浏览}
~Lablin~图片浏览软件为~fbv~,用~fbv~软件打开/root/g-bios.jpg图片。
\begin{lstlisting}[numbers=none]
# fbv /root/g-bios.jpg
\end{lstlisting}
效图如下图所示。

\subsection{音/视频播放}
还记得第一章中源码编译的~mpg123 mp3~播放器吗?~Lablin~中也有移植这个软件哦:)
\begin{lstlisting}[numbers=none]
# mpg123 /root/22.mp3
\end{lstlisting}

~Lablin~中移植了~mplayer~视频插放器。在/root的目录下存放了一个beer.mpg视频。一起感受一下:
\begin{lstlisting}[numbers=none]
# mplayer /root/berr.mpg
\end{lstlisting}
效图如下图所示。

RMVB?

\subsection{3D~游戏}
3D游戏doom。
\begin{lstlisting}[numbers=none]
# doom
\end{lstlisting}
\subsection{Wi-Fi}
\subsection{USB}
\subsection{SD~卡}

\section{Labin~API}

\ifx \inmaxwitbook \undefined
\end{CJK*}
\end{document}
\fi
